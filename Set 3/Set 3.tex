\documentclass[12pt]{article}
\usepackage[margin=1in]{geometry} 
\usepackage{amsthm}
\usepackage{amsmath}
\usepackage{amsfonts}
\usepackage{amssymb}
\usepackage{multicol}
\usepackage{graphicx}
\usepackage{float}
\usepackage{makecell}
\usepackage{fancyvrb} 
\newcommand{\verbatimfont}[1]{\renewcommand{\verbatim@font}{\ttfamily#1}}
\usepackage{siunitx}
\usepackage{blkarray} 
\usepackage{hyperref}
\usepackage{soul}
\usepackage{titling}
\sethlcolor{lightgray}
\setlength{\columnsep}{1cm}
\DeclareMathOperator{\R}{\mathbb{R}}
\renewcommand\qedsymbol{$\blacksquare$}

\setlength\parindent{0pt} 

\title{Problem Set 3} 

\author{Jakub Gadawski\\ 
Optimization methods
}

\date{\today}
\begin{document}
\setlength{\droptitle}{-5em}
\maketitle

\section*{Exercise 1}
{\bfseries Check, if set is a cone:}
\subsection*{a) \(A=\{x \in \R^n : x_i \geq 0\} \)}
The set is a cone.
\begin{proof}
    We have to check, if \(\forall_{x\in A}\forall_{\alpha \in \R, \alpha \geq 0} \alpha x \in A\)\\
    Let \(x\in A, x=(x_1, x_2, \ldots, x_n), \alpha \in \R, \alpha \geq 0\)\\
    Then, \( \alpha x = (\overbrace{\alpha x_1}^{R_+}, \overbrace{\alpha x_2}^{R_+}, \ldots, \overbrace{\alpha x_n}^{R_+}) \in A \)
\end{proof}
\subsection*{c) \( C=\{ x \in \R^n:<x, a> \leq 0\}, a\in \R^n\)}
The set is a cone.
\begin{proof}
    Let \( x \in C, \alpha \in \R, \alpha \geq 0\)\\
    Then \( <\alpha x, a>=\alpha <x, a> \leq 0 \implies \alpha x \in C \)
\end{proof}
\subsection*{d) \(D=(a):=\{x \in \R^n : x=\alpha a, \alpha \geq 0\}, a \in \R^n\) }
The set is a cone.
\begin{proof}
    Let \( x \in D, \alpha \in \R, \alpha \geq 0 \)\\
    Then \(\alpha x \in D\) (definition of D)
\end{proof}
\subsection*{e) \(E=\{x \in \R^n : x=\alpha a, \alpha > 0\}, a \in \R^n\) }
The set is a cone for \(a=0\).\\
If \(\alpha = 0 \land a \ne 0 \implies \alpha x \notin E \implies E\,\)- not a cone.
\subsection*{f) \(F=\{x \in \R^2 : x_1=2\}\) }
The set is a cone.\\
If \(\alpha = 2 \land x = (2,4) \implies \alpha x=(4, 8) \notin F\implies F\,\)- not a cone.
\section*{Exercise 2}
{\bfseries Prove, that:}
\subsection*{a) \( C_1, C_2 - cones \implies C_1 \cap C_2 - cone\)}
\begin{proof}
    Let \( x \in C_1 \cap C_2\)\\
    Then \( x \in C_1 \land x \in C_2 \) \\
    But \( C_1, C_2 - cones \implies \forall_{\alpha \geq 0} (x \alpha \in C_1 \land x \alpha \in C_2) \implies \\ \implies \forall_{\alpha \geq 0} x\alpha \in C_1 \cap C_2 \implies C_1 \cap C_2 - cone\) 
\end{proof}
If sets are convex, then intersection is a convex set.
\subsection*{b) \(S_1, S_2 - cones \implies S_1 \cup S_2 - cone\) }
\begin{proof}
    Let \( x \in S_1 \cup S_2\)\\
    Then \( x \in S_1 \lor x \in S_2 \)\\
    If \( x \in S_1 \implies \forall_{\alpha \geq 0} \alpha x \in S_1 \implies S_1 \cup S_2\)\\
    If \( x \in S_2 \implies \forall_{\alpha \geq 0} \alpha x \in S_2 \implies S_1 \cup S_2\)\\
    So \( S_1 \cup S_2 - cone\)
\end{proof}
Let \( S_1 = \{x: \alpha[1,1], \alpha \geq 0\}, S_2 = \{x: \alpha[-1,1], \alpha \geq 0\} \)\\
Then \( S_1, S_2\) - convex cones, but point \((0, 1)\) is in segment \( ((-1, 1), (1,1)) \) and \( (0,1) \notin S_1 \cup S_2 \implies S_1 \cup S_2 - not convex set \)
\subsection*{c) \( S_1, S_2 -cones \implies S_1+S_2 -cone\)}
\begin{proof}
Let \(x \in S_1 +S_2, x=x_1+x_2, x_1 \in S_1, x_2 \in S_2\)\\
Then \( \forall_{\alpha \geq 0} \alpha x_1 \in S_1, \alpha x_2 \in S_2  \)\\
And \( \forall_{\alpha \geq 0} \alpha x_1+\alpha x_2 \iff \forall_{\alpha \geq 0} \alpha x\)\\
And all x could be presented as \( x_1+x_2, x_1 \in S_1, x_2 \in S_2\)
\end{proof}
For all convex sets \(S_1, S_2, S_1+S_2 -convex\)
\subsection*{d) \( S - cone \implies -S - cone\)}
\begin{proof}
    Let \(x \in -S\)\\
    Then \( \forall_{\alpha \geq 0} \alpha(-x)=-\alpha x, \alpha x \in S \implies -S -cone\)
\end{proof}
\subsection*{e) \( S \subset \R^n - cone, f:\R^n \mapsto \R^m -linear  \implies f(S)- cone\)}
\begin{proof}
    Let \(x \in f(S), \alpha \geq 0, x=f(x_1), x_1 \in S\)\\
    Because any linear tramsformation could be represented as a matrix  \(f=A, A \in M_{m \times n}\)\\
    Then \(\alpha x=\alpha f(x_1)=\alpha A x_1 =A \alpha x_1= f(\alpha x_1)\)\\
    and \(\alpha x \in S\), because \(S - cone \implies f(S)-cone\)
\end{proof}
%\subsection*{f) \( F=\{ y \in \R^m : y=Ax, x \geq0, x \in \R^n\} - convex\,cone\)}
%\begin{proof}
%    Set \({ x \in \R^n: x \geq0\}\) is a cone, and matrix A is a linear transformation, so F is a cone.\\
    
%\end{proof}
\section*{Exercise 3}
{\bfseries Give geometric interpretation of cone and conjugated cone:}
\subsection*{c) \(D=([2,-1, 6]^T)\)}
It's half line set down by \([2, -1.6]\), conjugated cone is \(D^*=\{(x, y, z):2x-y+6z \leq 0\}\)
\subsection*{d) \(E=\{x \in \R^3: x_i\geq 0 \}\)}
Conjugated cone to E is \( E^*=\{x \in \R^3: x_i\leq 0 \} \)
\subsection*{e) \(F=\{x \in \R^2: \begin{bmatrix}1&2\\2&-1\end{bmatrix} x\geq 0 \}\)}
\begin{math}
    \begin{bmatrix}
    1&2\\2&-1
    \end{bmatrix} 
    \begin{bmatrix} 
    x \\ y 
    \end{bmatrix} \geq 0\\
    y\geq \frac{-x}{2} \land y \leq 2x\\$ $\\
    F=\{x \in \R^2: x_1 \in R \land  \frac{-x_1}{2} \leq x_2 \leq 2x_1  \}\\
    F^*=\{x \in \R^2: x_1 \in R \land  \frac{-x_1}{2} \geq x_2 \geq 2x_1  \}
\end{math}

\end{document}