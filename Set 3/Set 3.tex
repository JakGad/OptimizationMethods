\documentclass[12pt]{article}
\usepackage[margin=1in]{geometry} 
\usepackage{amsthm}
\usepackage{amsmath}
\usepackage{amsfonts}
\usepackage{amssymb}
\usepackage{multicol}
\usepackage{graphicx}
\usepackage{float}
\usepackage{makecell}
\usepackage{fancyvrb} 
\newcommand{\verbatimfont}[1]{\renewcommand{\verbatim@font}{\ttfamily#1}}
\usepackage{siunitx}
\usepackage{blkarray} 
\usepackage{hyperref}
\usepackage{soul}
\usepackage{titling}
\sethlcolor{lightgray}
\setlength{\columnsep}{1cm}
\DeclareMathOperator{\R}{\mathbb{R}}
\renewcommand\qedsymbol{$\blacksquare$}

\setlength\parindent{0pt} 

\title{Problem Set 3} 

\author{Jakub Gadawski\\ 
Optimization methods
}

\date{\today}
\begin{document}
\setlength{\droptitle}{-5em}
\maketitle

\section*{Exercise 1}
{\bfseries Check, if set is a cone:}
\subsection*{a) \(A=\{x \in \R^n : x_i \geq 0\} \)}
The set is a cone.
\begin{proof}
    We have to check, if \(\forall_{x\in A}\forall_{\alpha \in \R, \alpha \geq 0} \alpha x \in A\)\\
    Let \(x\in A, x=(x_1, x_2, \ldots, x_n), \alpha \in \R, \alpha \geq 0\)\\
    Then, \( \alpha x = (\overbrace{\alpha x_1}^{R_+}, \overbrace{\alpha x_2}^{R_+}, \ldots, \overbrace{\alpha x_n}^{R_+}) \in A \)
\end{proof}
\subsection*{c) \( C=\{ x \in \R^n:<x, a> \leq 0\}, a\in \R^n\)}
The set is a cone.
\begin{proof}
    Let \( x \in C, \alpha \in \R, \alpha \geq 0\)\\
    Then \( <\alpha x, a>=\alpha <x, a> \leq 0 \implies \alpha x \in C \)
\end{proof}
\subsection*{d) \(D=(a):=\{x \in \R^n : x=\alpha a, \alpha \geq 0\}, a \in \R^n\) }
The set is a cone. 
\begin{proof}
    Let \( x \in D, \alpha \in \R, \alpha \geq 0 \)\\
    Then \(\alpha x \in D\) (definition of D)
\end{proof}
\subsection*{e) \(E=\{x \in \R^n : x=\alpha a, \alpha > 0\}, a \in \R^n\) }
The set is a cone for \(a=0\).\\
If \(\alpha = 0 \land a \ne 0 \implies \alpha x \notin E \implies E\,\)- not a cone.
\subsection*{f) \(F=\{x \in \R^2 : x_1=2\}\) }
The set is a cone.\\
If \(\alpha = 2 \land x = (2,4) \implies \alpha x=(4, 8) \notin F\implies F\,\)- not a cone.

\end{document}